\section{Future Work}
Implementing a recommendation engine is not an easy task.  Future work
includes implementation of our situational recommendation engine, and
tweaking the voting parameters such that the recommended programs are
consistent with what users like.  Each of the factors we proposed must
be given a certain weight in the recommendation algorithm, and by
getting more user data, we can start to figure out what the
appropriate weights should be.  We can also couple machine learning
algorithms with user feedback to find patterns in user’s viewing
habits.

In addition, {\sys} currently exists as a stand alone website.  {\sys}
could also exist as an app for a set top box, such as AppleTV, Google
TV, Boxee, or Roku.  Given a user interface makeover, {\sys} could
also exist as a second screen experience.

{\sys} could also benefit with tighter integration with Facebook’s
social graph.  Like Hulu and Spotify, it could utilize the ticker, as
well as look at user’s network and friends to make assumptions about
the viewer’s watching habits.

In terms of content acquisition, the current television model must be
changed to allow for greater freedom from traditional cable
distribution methods \cite{montpetit}.  {\sys} will stand as a layer
on top of existing television content providers, such as Hulu or
Netflix.  {\sys}, from the business perspective, will use paid
promoted suggestions and inline advertising to create revenue.  In
addition, the data analytics collected by our system could be used to
generate revenue.

\section{Conclusion}

{\sys} is an application which helps users keep track of shows they
want to watch, and uses individual user preferences and watching
habits to help give tv show suggestions for groups.  {\sys} let's
users search for shows based on recommendations from collaborative
filtering and specified moods.  We have implemented {\sys} as a stand
alone web prototype.

%% NEHA TODO

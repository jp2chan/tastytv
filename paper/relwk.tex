\section{Related Work}

\subsection{Social Television}
The current sphere of social television mostly exists online.  There’s
buzz over social networks like twitter or facebook, blogposts, online
communities, or even virtual watching rooms.  Nonetheless, all of
these outlets emphasize the virtual presence of viewers.  None of
these services aim at bringing users together in real life, rather
they encourage exchange over the internet.  A majority of social tv
(about 95\%) is on social platforms like twitter and
facebook~\cite{chausse}.  However, many existing apps allow users to
chat about different shows.  This includes BeeTV, HashTV, TV Chatter,
TVTweet, CommenTV, and TweetYourTV~\cite{chausse}.  Again, all of
these applications only encourage online interaction.

\subsection{Recommender Systems} 
\label{sec:recommender}
There are three different main approaches to creating a recommendation
system: collaborative filtering, content filtering, and a hybrid
approach~\cite{su,vozalis}.  The first method, collaborative
filtering, relies on nearest neighbor relationships and item/user
based recommendations \cite{su}.  This method is most predominantly
used by Amazon and Last.fm.  However, traditional collaborative
filtering suffers from several problems.  The first problem is data
sparsity~\cite{su,melville}, which it tries to alleviate by
dimensionality reduction, using the SVD and Principal Component
Analysis to extract the key features.  Nonetheless, it also suffers
from scalability problems~\cite{papagelis}, synonymy, the grey sheep
syndrome (where a person exhibits inconsistent behavior)~\cite{su},
and shiling attacks (people trying to promote their own material).
The second method, content filtering, adopted by systems like Pandora
and Rotten Tomatoes, uses Bayesian belief nets and clustering to
recommend items based upon discrete classifiers on the content.
Finally, the hybrid approach combines both collaborative and content
filtering, which is what Netflix does~\cite{su,melville}.  All
conclude that the hybrid approach, although more complex, increases
prediction performance.  Thus, {\sys} adopts a hybrid of collaborative
and content filtering, putting a large emphasis on your real life
social tv watching habits, as well as adding other potential factors
to the recommendation voting.

\subsection{Internet Radio}
Internet radio, in terms of socialization and personalization, has
evolved much quicker than television.  Thus, {\sys} draws inspiration
from the models that internet radio has created.  The first
inspiration is the “currently listening” queue that services like
Grooveshark~\cite{grooveshark} and Spotify~\cite{spotify} have.  They
make a distinction between the user’s currently streaming queue and
the list of songs that the user may be interested in.  In addition,
Pandora’s~\cite{pandora} content-based collaborative filtering system
inspired {\sys}.  In Pandora, songs are tagged with specific details
to help create better recommendations.

\subsection{Related Work}
Hulu is the first major television outlet to incorporate directly with
facebook and the ticker, similar to Spotify.  Hulu allows you to see
shows that your friends are watching, and broadcasts the shows that
you watch to your friends.  It aggregates the shows that you watch, as
well as lets you have conversations with your friends.

There are also many apps that allow users to “check in” to tv shows
and see what they’re friends are watching and have watched, including
GetGlue, IntoNow, ScreenTribe, TunerFish, and WatchPoints~\cite{chausse}.  
However, most of these never involve real life user
interaction.  IntoNow notifies your friend if you both are watching at
the same time, but it’s retroactive socialization in that it notifies
you after you’ve already made the conscious effort to already start
watching the show.  On the other hand, {\sys} pushes for proactive
socialization in that you’re already hanging out with friends and want
to decide what to watch.  {\sys} encourages viewers to watch things
with their friends in real life.
